\documentclass[a4paper,14pt]{article}
\usepackage[T2A]{fontenc}			      % кодировка
\usepackage[utf8]{inputenc}               % кодировка исходного текста
\usepackage[english, russian]{babel}   % локализация и переносы


%%% Страница 
\usepackage{extsizes} % Возможность сделать 14-й шрифт
\usepackage{geometry}  
\geometry{left=20mm,right=20mm,top=25mm,bottom=30mm} % задание полей текста


%%%  Текст
\setlength\parindent{0pt}         % Устанавливает длину красной строки 0pt
\sloppy                                        % строго соблюдать границы текста
\linespread{1.3}                           % коэффициент межстрочного интервала
\setlength{\parskip}{0.5em}                % вертик. интервал между абзацами
%\setcounter{secnumdepth}{0}                % отключение нумерации разделов
\usepackage{multicol}				          % Для текста в нескольких колонках
%\usepackage{soul}
\usepackage{soulutf8} % Модификаторы начертания


%%% Гиппер ссылки
\usepackage{hyperref}
\usepackage[usenames,dvipsnames,svgnames,table,rgb]{xcolor}
\hypersetup{				% Гиперссылки
	unicode=true,           % русские буквы в раздела PDF\\
	pdfstartview=FitH,
	pdftitle={Заголовок},   % Заголовок
	pdfauthor={Автор},      % Автор
	pdfsubject={Тема},      % Тема
	pdfcreator={Создатель}, % Создатель
	pdfproducer={Производитель}, % Производитель
	pdfkeywords={keyword1} {key2} {key3}, % Ключевые слова
	colorlinks=true,       	% false: ссылки в рамках; true: цветные ссылки
	linkcolor=blue,          % внутренние ссылки
	citecolor=green,        % на библиографию
	filecolor=magenta,      % на файлы
	urlcolor=NavyBlue,           % на URL
}


%%% Для формул
\usepackage{amsmath}          
\usepackage{amssymb}


\usepackage{amsthm}  % for theoremstyle

\theoremstyle{plain} % Это стиль по умолчанию, его можно не переопределять.
\newtheorem*{theorem}{Теорема}
\newtheorem*{prop}{Утверждение}
\newtheorem*{lemma}{Лемма}
\newtheorem*{sug}{Предположение}

\theoremstyle{definition} % "Определение"
\newtheorem*{Def}{Определение}
\newtheorem*{corollary}{Следствие}
\newtheorem{problem}{Задача}[section]

\theoremstyle{remark} % "Примечание"
\newtheorem*{nonum}{Решение}
\newtheorem*{defenition}{Def}
\newtheorem*{example}{Пример}
\newtheorem*{note}{Замечание}


%%% Работа с картинками
\usepackage{graphicx}                           % Для вставки рисунков
\graphicspath{{images/}{images2/}}        % папки с картинками
\setlength\fboxsep{3pt}                    % Отступ рамки \fbox{} от рисунка
\setlength\fboxrule{1pt}                    % Толщина линий рамки \fbox{}
\usepackage{wrapfig}                     % Обтекание рисунков текстом
\graphicspath{{images/}}                     % Путь к папке с картинками


%%% облегчение математических обозначений


\newcommand{\all}{\forall}
\newcommand{\ex}{\exists}

\newcommand{\RR}{\mathbb{R}}
\newcommand{\NN}{\mathbb{N}}
%\newcommand{\C}{\mathbb{C}}             % команда уже определена где-то)
\newcommand{\ZZ}{\mathbb{Z}}
\newcommand{\EE}{\mathbb{E}}
\newcommand{\brackets}[1]{\left({#1}\right)}      % автоматический размер скобочек
% Здесь можно добавить ваши индивидуальные сокращения  

\makeatletter
\def\@seccntformat#1{\@ifundefined{#1@cntformat}%
   {\csname the#1\endcsname\space}%    default
   {\csname #1@cntformat\endcsname}}%  enable individual control
\def\section@cntformat{\thesection.\space} % section-level
\makeatother

\usepackage{listings}
\usepackage{xcolor}

\lstset{
  language=C++,
  basicstyle=\ttfamily\small,
  keywordstyle=\color{blue},
  commentstyle=\color{gray},
  numbers=left,
  numberstyle=\tiny,
  stepnumber=1,
  numbersep=5pt,
  tabsize=2,
  breaklines=true,
  showstringspaces=false
}


\title{Программа к экзамену по курсу "Алгоритмы и структуры данных".}
\date{}

\begin{document}
    \maketitle
    \newpage
    \tableofcontents
    \newpage
    \section{Вопросы на неуд.}
    
    \subsection{Понятие графа. Способы хранения графа: список смежности, матрица смежности, список ребер.}

    \begin{Def}
        Ориентированный граф(далее орграф) - это $G = (V,E)$, где $V$ - это множество вершин, а $E\subset V\times V$ - множество ребер.
    \end{Def}
    \begin{Def}
        Неориентированный граф - это $G = (V,E)$, где $V$ - это множество вершин, а $E=\{\{u,v\}: u,v \in V\}$ - множество ребер, причем $(u,v)\in E \Leftrightarrow (v,u)\in E$.
    \end{Def}

    \subsection{Отношение сильной связности.}
    \label{subsec:strongly_connected_components}
    \begin{Def}
    Две вершины сильно связны, если между ними есть путь в обе стороны.
    \end{Def}
    Отношение сильной связности является отношением эквивалентности (рефлексивное, симметричное, транзитивное).
    \begin{Def}
        Компоненты сильной связности (КСС) - классы эквивалентности по отношению сильной связности.
    \end{Def}
    \begin{Def}
        Компонента сильной связности - максимальное подмножество вершин, каждая из которых достижима из любой другой.
    \end{Def}
    \begin{Def}
        Графом конденсации называют граф, где все компоненты сильной
        связности сжаты до одной вершины, а ребра между ними
        получаются как ребра между компонентами.
    \end{Def}
    \subsubsection*{Алгоритм Косарайю.}
    \begin{enumerate}
        \item Запускаем DFS на графе, получить вершины в порядке
        увеличения времени выхода (почти топологическая
        сортировка).
        \item Транспонируем граф.
        \item Запустить на транспонированном графе DFS в порядке
        уменьшения времени выхода в исходном графе. Каждая
        найденная компонента является компонентой сильной
        связности.
    \end{enumerate}
    \begin{theorem}
        Алгоритм Косарайю корректен.
    \end{theorem}
    \begin{proof}
        То есть нам надо доказать, что на шаге 3 каждый запуск посетит
одну компненту сильной связности и только ее. Заметим, что в
транспонированном графе компоненты все те же, то есть мы
изменили лишь связи между компонентами. \\
Рассмотрим первый вызов DFS на третьем шаге. Так как это
вершина с максимальным временем выхода, то нет ребер в ее
компоненту сильной связности (по лемме выше). При этом в
транспонированном графе получаем, что из рассматриваемой
компоненты нет ребер в другие, а значит DFS посетит только саму
компоненту. Ну это именно то, что нам требовалось показать, так
как с помощью массива used мы отделили всю компоненту. \\
Рассуждая по индукции, получим, что мы генерируем компоненты
сильной связности, при этом в порядке топологической сортировки,
так как мы все еще идем по убыванию времени выхода. \\
    \end{proof}

    \subsection{Обход в глубину \texttt{DFS}. Атрибуты вершин: времена входа и выхода, цвета. Топологическая сортировка. Классификация ребер в обходе \texttt{DFS}.}

    \subsubsection*{Цвета вершин.}
    \begin{itemize}
        \item Белый - не посещена.
        \item Серый - посещена, но не все соседи рассмотрены.
        \item Черный - посещена, все соседи рассмотрены.
    \end{itemize}

    \subsubsection*{DFS.}
    \begin{enumerate}
        \item Красим в серый.
        \item Просматриваем соседей.
        \item Если сосед белый - вызываем рекурсивно от него.
        \item Рассмотрели всех соседей - красим в черный.
    \end{enumerate}
    \texttt{Время: } $O(|V|+|E|)$

    \subsubsection*{Классификация ребер.}
    \begin{itemize}
        \item Ребра деревьев обхода (tree edge) — ребро $(u,v)$ является
        ребром дерева, если при исследовании ребра была впервые
        открыта вершина $v$.
        \item Обратные ребра (back) — это ребра $(u,v)$, соединяющие
        вершину $u$ с ее предком $v$ в дереве поиска в глубину. Петли
        тоже считаем обратными ребрами.
        \item Прямые ребра (forward) — это ребра $(u,v)$, не являющиеся
        ребрами дерева и соединяющие вершину $u$ с ее потомком $v$ в
        дереве поиска в глубину.
        \item Перекрестные ребра (cross) — все остальные ребра, они могут
        соединять вершины одного и того же дерева поиска в глубину,
        когда ни одна из вершин не является предком другой, или
        соединять вершины в разных деревьях поиска в глубину.
    \end{itemize}
    \subsubsection*{Топологическая сортировка.}
    \begin{itemize}
        \item $time$ -- время.
        \item $tin_v$ -- время входа (серый).
        \item $tout_v$ -- время выхода (черный).
    \end{itemize}

    \begin{enumerate}
        \item Используем DFS.
        \item При выходе из вершины вносим ее в начало списка.
    \end{enumerate}
    
    \subsection{Отношение реберной двусвязности. Мосты.}
    \label{subsec:edge_biconnectivity}
    \begin{Def}
        Деревом обхода DFS называют граф, состоящий из вершин,
посещаемых в ходе обхода DFS и следующих ребер:
        \begin{itemize}
            \item Древесное ребро — ребро, по которому DFS переходит напрямую
            (переходы в белые вершины из серых);
            \item Обратное ребро — ребро, которое DFS просматривает, но не идет
            по нему (переходы в серые вершины).
        \end{itemize}
    \end{Def}
    Введем функцию tup(v), определяемую следующим образом:
    \[
    \text{tup}(v) = \min \left\{
    \begin{array}{l}
    \text{tin}(v), \\[6pt]
    \min\limits_{u} \text{tin}(u)
    \end{array}
    \right.
    \]
    Где $u$ — предок $v$ и при этом $u$ достижима по обратному ребру из $w$ — вершины поддерева $v$.

    \begin{Def}
        Две вершины реберно двусвязны, если между ними есть два реберно непересекающихся пути.
    \end{Def}
    \begin{prop}
        Несложно заметить, что отношение реберной двусвязности является отношением эквивалентности.
    \end{prop}
    \begin{Def}
        Компонентой реберной двусвязности называют класс эквивалентности по отношению выше.
    \end{Def}
    \begin{Def}
        Мост — ребро, при удалении которого увеличивается число компонент связности.
    \end{Def}

    \begin{theorem}[Критерий моста]
        Древесное ребро $(t, v)$ является мостом если и только если $t_{up}(v) \geqslant t_{in}(v)$.    
    \end{theorem}
    \begin{proof}
        $t_{up}(v) \leqslant t_{in}(v)$ по определению. Осталось
рассмотреть случай равенства $t_{up}(v) = t_{in}(v)$.
Это равносильно тому, что не найдется вершины $u$, в которую по
обратному ребру можно прыгнуть из поддерева $v$ выше, чем $v$ (так
как иначе это было бы $t_{in}(u) < t_{in}(v)$), что то же самое, что ребро
является мостом.
    \end{proof}

    \subsection{Взвешенные графы. Обход в ширину \texttt{BFS}.}
    \label{subsec:bfs}
    \begin{Def}
        Взвешенным графом будем называть тройку $G = (V, E,w)$, где $V$ и $E$ уже привычные нам составляющие, а вот $w: E \rightarrow K \subseteq R$ - весовая функция.
    \end{Def}
    \begin{problem}
        Найти расстояния от $s$ до всех остальных, если $w : E \rightarrow K = \{1\}$.
    \end{problem}

    \subsubsection*{Обход в ширину BFS.}
    \begin{enumerate}
        \item Заведем очередь. Положим туда $s$.
        \item Извлечем вершину из очереди, пометим как посещенную, добавим смежные.
        \item Пока очередь не пуста - повторяем шаг 2 c одним условием: если
        извлеченная вершина уже посещена, то не добавляем ее в
        очередь.
    \end{enumerate}
    \subsubsection*{0-1 BFS.}
    Вместо очереди заведем дек. Если $w(from, to)=0$ -- в начало, иначе -- в конец.
    \subsubsection*{1-k BFS.}
    Массив очередей. \\
    Время: $O(k|V|+|E|)$. Память: $O(k|V|+|E|)$.
    \subsubsection*{Оптимизация памяти 1-k BFS.}
    На самом деле можно заметить, что
достаточно использовать только $k + 1$ очередей, так как только $k$
очередей непусты единовременно. Поэтому надо обновлять расстояния
до вершин и писать в $at\_dist[(d + w) \%(k + 1)]$.

    \subsubsection*{0-k BFS.}
    Как предыдущий, только вместо очередей - деки.

    \subsection{Задачи, решаемые алгоритмами Дейкстры и Форда-Беллмана. Время работы.}
    
    \subsubsection*{Алгоритм Дейкстры.}
    \label{subsubsec:decstra}
    \begin{problem}
        Дан взвешенный граф такой, что $w: E\rightarrow \RR^+$ и зафиксирована вершина $s \in V$. Нужно найти $\forall v \in V \ dist(s,v)$.
    \end{problem}

        Опишем алгоритм Дейкстры, решающий данную задачу. \\
    Инициируем множество $S = \{s\}$ — множество вершин, для которых
    кратчайшее расстояние вычислено корректно на текущий момент
    времени, также будет массив $d[]$ текущих оценок на вес
    кратчайшего пути до вершин.\\
    Очевидно, $d[s] = 0$, для $v \in V \setminus S$  $d[v] = \infty$. \\
    Далее повторяем следующий алгоритм:
    \begin{enumerate}
        \item Рассмотрим все вершины v такие, что $v \notin S$, выберем среди
        них такую, что $d[v]$ минимально.
        \item Добавим $v$ в множество $S$, присвоим $dist(s,v) = d[v]$ (докажем
        ниже).
        \item Рассмотрим ребра вида $(v,t)$, запишем
        $d[t] = \min(dist(s,v)+w(v,t),d[t])$, то есть проведем релаксацию.
        \item Пока $S \neq V$, то повтори шаги выше.
    \end{enumerate}
    \subsubsection*{Время работы алгоритма Дейкстры.}
    \begin{itemize}
        \item Уменьшение оценки $d$ для вершины. Каждое ребро уменьшает
        не более одного раза, значит таких операций $O(|E|)$.
        \item Получение вершины с минимальной оценкой $d$ не из $S$.
        Каждая вершина извлекается не более одного раза, а значит
        таких операций $O(|V|)$.
    \end{itemize}
    \begin{table}[h!]
        \centering
        \begin{tabular}{|c|c|c|c|}
        \hline
        \textbf{Контейнер вершин} & \textbf{Релаксация} & \textbf{Извлечение} & \textbf{Итого} \\
        \hline
        Массив & $O(1)$ & $O(|V|)$ & $O(|V|^2)$ \\
        \hline
        Дерево поиска & $O(\log |V|)$ & $O(\log |V|)$ & $O(|E| \log |V|)$ \\
        \hline
        Фибоначчиева куча & $O^*(1)$ & $O(\log |V|)$ & $O(|E| + |V| \log |V|)$ \\
        \hline
        \end{tabular}
    \end{table}        

    \subsubsection*{Алгоритм Форда-Беллмана.}
    \label{subsubsec:ford_bellman}
    \begin{problem}
        Найти расстояния от s до всех остальных, если $w: E\rightarrow \RR$. Считаем, что циклов отрицательного веса нет.
    \end{problem}
    \begin{enumerate}
        \item Пусть $dp[v][k]$ равно минимальному весу пути из $s$ в $v$,
        состоящему из ровно $k$ ребер.
        \item База. $dp[s][0] = 0, dp[:][:] = \infty$.
        \item Переход. $dp[v][k] = \min\limits_{(u,v)\in E}(dp[u][k-1]+w(u,v))$.
        \item Порядок пересчета. Внешний цикл по $k$, внутри цикл по
        ребрам.
        \item Ответ. $ans[v] = \min\limits_k dp[v][k]$.
    \end{enumerate}
    Время: $O(|V||E|)$. Память: $O(|V|^2)$.

    \subsubsection*{Поиск циклов отрицательного веса.}
    \label{subsubsec:ford_bellman_negative_cycle}
    \begin{problem}
        Циклом отрицательного веса назовем цикл $v_1, v_2, \ldots, v_n, v_1$, у которого $\sum\limits_{i=1}^{n} w(v_i, v_{(i+1)\%n}) < 0$.
    \end{problem}
    Адаптируем алгоритм Форда-Беллмана так, чтобы он хранил для каждой вершины еще предка, из которого она релаксировалась.
    \begin{prop}
        На $|V|$-й итерации найдется вершина v, до которой расстояние
    уменьшилось по сравнению с $(|V| - 1)$-й итерацией тогда и только
    тогда, когда в графе есть цикл отрицательного веса, достижимый
    из $s$.
    \end{prop}
    \begin{proof}
        $\Rightarrow$ Простой кратчайший путь не может быть длиннее $|V|-1$
        ребер, а если произошла релаксация, то существует не простой
        путь, имеющий вес строго меньший. А значит есть цикл
        отрицательного веса. \\
        $\Leftarrow$ Рассмотрим цикл отрицательного веса $C = c_1, c_2, \ldots, c_k$. Так как
        $|C| < |V|$, на $|V|$-й итерации $\exists i : c_i$ будет рассмотрена второй
        раз, при этом она будет рассмотрена по пути вдоль
        отрицательного цикла, а значит произойдет релаксация.
    \end{proof}
    
    \subsection{Постановка задач по поиску минимального остовного дерева. Время ее решения алгоритмом Прима или Крускала.}
    \begin{problem}
        Есть сеть, состоящая из $n$ городов, и мы хотим соединить города
    интернетом, чтобы такая сеть была связна, для этого достаточно
    чтобы полученный граф был деревом. В дереве из $n$ вершин у нас
    $n-1$ ребро.
    \end{problem}
    \begin{Def}
        Матроидом называется пара $(X, \mathcal{I})$, где $X$~-- множество элементов, называемое носителем матроида, а $\mathcal{I}$~-- некоторое множество подмножеств $X$, называемое семейством независимых множеств. В матроиде должны выполняться следующие свойства:  
        \begin{itemize}
        \item $\varnothing \in \mathcal{I}$ -- пустое множество независимо;
        \item если $A \subseteq B$, $B \in \mathcal{I}$, то $A \in \mathcal{I}$ (подмножество независимого множества -- независимо);
        \item если $A, B \in \mathcal{I}$ и $|A| < |B|$, то существует $x \in B \setminus A$ такой, что $A \cup \{x\} \in \mathcal{I}$.
        \end{itemize}
    \end{Def}
    \begin{Def}
        Матроид называется взвешенным, если на нем задана весовая функция: $\omega(A)=\sum \omega(a_i)$.
    \end{Def}
    \begin{theorem}[Радо-Эдмондса]
        Пусть $A \in \mathcal{I}$~-- множество минимального веса среди всех независимых подмножеств $X$ мощности $k$. Возьмём такой элемент $x \notin A$, что $A \cup \{x\} \in \mathcal{I}$ и вес $\omega(x)$ минимален. Тогда $A \cup \{x\}$~-- множество минимального веса среди независимых подмножеств $X$ мощности $k+1$.
    \end{theorem}
    \subsubsection*{Алгоритм Радо-Эдмондса.}
        Пусть нам нужно найти независимое множество, которое будет
    включать как можно больше элементов и при этом иметь как
    можно меньший вес. Из теоремы Радо-Эдмонса следует, что нам
    достаточно отсортировать элементы по возрастанию, и в таком
    порядке добавлять их в ответ.
    
    \subsubsection*{Алгоритм Крускала.}
    \label{subsec:kruskal}
    Применив теорему Радо-Эдмондса к примеру (множеству лесов графа), получим жадный алгоритм:
    \begin{enumerate}
        \item Сортируем ребра по весам от меньшего к большему.
        \item Заводим множество $T$, которое изначально пусто.
        \item Рассматриваем ребра по возрастанию веса: если добавление
        ребра в T не делает множетво T циклическим(граф с циклом),
        добавляем новое ребро в это множество.
    \end{enumerate}
    Основная проблема алгоритма – добавление элементов в множество $T$. \\
    Время $O(|E|\log |E|)$. Память $O(|V|)$.

    \subsubsection*{Алгоритм Прима.}
    \begin{Def}
        (S, T) - разрез, если $(S\cup T=V)\wedge (T\cap S=\emptyset)$
    \end{Def}
    \begin{Def}
        $(u,v)$ – пересекает разрез, если $u$ и $v$ лежат в разных частях разреза.
    \end{Def}
    \begin{Def}
        Пусть $G' = (V, E')$~-- подграф некоторого минимального остовного дерева $G$. Ребро $(u, v) \notin G'$ называется \textit{безопасным}, если при добавлении его в $G'$, то $G' \cup \{(u, v)\}$ также является подграфом некоторого минимального остовного дерева графа $G$.
    \end{Def}
    \begin{lemma}[О безопасном ребре.]
        Рассмотрим связный неориентированный взвешенный граф $G = (V, E)$ с весовой функцией $\omega : E \to \mathbb{R}$. Пусть $G' = (V, E')$~-- подграф некоторого минимального остовного дерева графа $G$, $\langle S, T \rangle$~-- разрез графа $G$ такой, что ни одно ребро из $G'$ не пересекает этот разрез, а $(u, v)$~-- ребро минимального веса среди всех рёбер, пересекающих разрез $\langle S, T \rangle$. Тогда ребро $e = (u, v)$ является \textit{безопасным} для $G'$.
    \end{lemma}
    \begin{proof}
        Достроим $E'$ до некоторого минимального остовного дерева, обозначим его $T_{\text{min}}$. Если ребро $e \in T_{\text{min}}$, то лемма доказана, поэтому рассмотрим случай, когда ребро $e \notin T_{\text{min}}$. Рассмотрим путь в $T_{\text{min}}$ от вершины $u$ до вершины $v$. Так как эти вершины принадлежат разным долям разреза, то хотя бы одно ребро пути пересекает разрез, назовём его $e'$. По условию леммы $w(e) \leq w(e')$.

        Заменим ребро $e'$ в $T_{\text{min}}$ на ребро $e$. Полученное дерево также является минимальным остовным деревом графа $G$, поскольку все вершины $G$ по-прежнему связаны и вес дерева не увеличился. Следовательно, $E' \cup \{e\}$ можно дополнить до минимального остовного дерева в графе $G$, то есть ребро $e$~-- безопасное.
    \end{proof}
    Алгоритм Прима:
    \begin{enumerate}
        \item Инициируем множество $S = \{s\}$ – множество вершин, на
        которых уже построен миностов.
        \item Рассмотрим разрез $\langle S, T \rangle$. Найдем безопасное для него ребро
        $e = (u,v)$, где $u \in S$.
        \item Добавим $e$ в миностов и $v$ в $S$.
        \item Добавим в множество ребер, пересекающих разрез ребра,
        выходящие из $v$ и идущие не в $S$.
        \item Пока $S \neq V$, повтори шаги выше.
    \end{enumerate}

    \subsubsection*{Время работы алгоритма Прима.}
    \begin{itemize}
        \item Временная сложность при использовании массива — $O(|V|^2)$
        \item Временная сложность при использовании бинарной
        кучи/дерева — $O(|E|\log |V|)$
        \item Временная сложность при использовании фибоначчиевой кучи
        — $O(|E|+|V|\log |V|)$
    \end{itemize}


    \subsection{Система непересекающихся множеств: API, наивная реализация на массиве.}
    \label{subsec:DSU}
        Заметим, что в процессе работы алгоритма Крускала компоненты связности графа можно представить как набор множетв, которые необходимо
    объединять, также проверять в одном ли множестве вершины.
    Хотим соорудить структуру, которая умеет эффективно выполнять
    следующие операции:
    \begin{itemize}
        \item Создать систему из $n$ множеств.
        \item Объединить два множества в одно.
        \item Проверить для двух элементов в одном или в разных
        множествax они лежат.
        \item Этого хватит для реализации алгоритма Крускала.
    \end{itemize}
    Такая структура данных носит название СНМ – cистема непересекающихся множеств.
    \subsubsection*{Наивная реализация.}
    \begin{lstlisting}
// Disjoint Set Union (DSU)
void MakeDSU(int cnt) {
    for (size_t id = 0; id < cnt; ++id) {
    parent[id] = id;
    }
}

int FindSet(int a_id) {
    if (a_id == parent[a_id]) {
    return a_id; // root
    }
    return FindSet(parent[a_id]);
}

void UnionSets(int a_id, int b_id) {
    a_id = FindSet(a_id);
    b_id = FindSet(b_id);
    if (a_id != b_id) {
    parent[a_id] = b_id;
    }
}
    \end{lstlisting}
    \subsubsection*{Эвристика сжатия пути.}
        Заметим что в функции FindSet мы можем сохранять
    возвращенный путь как предка, это значительно ускорит
    дальнейшие обращения.
    \subsubsection*{Эвристика объединения по рангу.}
    Будем помнить размеры деревьев. К большему по размеру дереву будем подвешивать меньшее.
    \subsubsection*{Время работы.}
    При объединении двух эвристик время работы на один запрос будет 
    составлять $O(\alpha (N))$, где $\alpha$ – обратная функция Аккермана. (Б/Д)

    \subsection{Постановка задачи поиска наименьшего общего предка. Наивное решение.}
    \label{subsec:lca}
  
    \begin{Def}
        Пусть дано дерево $T$, подвешенное за вершину $r$. Тогда назовем
    наименьшим общим предком двух вершин $u,v$ такую вершину $X$,
    что она лежит одновременно на путях $u \rightarrow r$ и $v \rightarrow r$, при этом
    такая вершина глубже всех таких вершин. \\
    Обозначение: $X = LCA(u,v)$.
    \end{Def}
    \subsubsection*{Наивное решение.}
    \begin{enumerate}
        \item Найдем глубину двух вершин.
        \item Для той вершины, которая лежит глубже поднимемся на
        разность высот.
        \item Далее будем подниматься до тех пор, пока не встретимся.
    \end{enumerate}
    Предпосчет: $O(|V|)$, запрос: $O(|V|)$.

    \subsubsection*{Двоичные подъемы.}
    \begin{itemize}
        \item Перед выполнением предпосчета посчитаем массивы: $parent_v$ и
        массив глубин $d_v$.
        \item Давайте предпосчитаем подъемы вверх, на расстояния степеней
        двойки.
        \item $dp[v][i]$ -- вершина, в которую мы попадем если поднимемся на $2^i$ шагов вверх.
        \item \[dp[v][i]=\left\{
        \begin{array}{l}
            parent[v], \ i=0, \\
            dp[dp[v][i-1]][i-1], \ i>0.
        \end{array}
        \right.\]
        \item Заведем два указателя, один на одну вершину, другой на вторую.
        \item Разложим разность глубин (вот для чего мы считали $d_v$) на
        сумму степеней двойки и поднимемся из более глубокой
        вершины до уровня менее глубокой вершины. 
        \item Теперь заметим, что если взять $k = \log_2 |V|$, можно уменьшать
        $k$ до тех пор, пока попадаем в одну вершину, если не попадаем,
        то делаем подъем для обеих. Продолжаем такие прыжки, пока
        можем уменьшать $k$.
        \item В итоге, когда мы остановимся, возьмем родителя любого из
        указателей.
    \end{itemize}
    Предпосчет: $O(|V|\log |V|)$, запрос: $O(\log |V|)$.

    \subsubsection*{Сведение LCA к RMQ.}
    \begin{itemize}
        \item Рассмотрим $u, v, w$, т.ч. $w= LCA(u, v)$.
        \item Рассмотрим обход dfs, запущенный из корня.
        \item Тогда сначала он посетит $w$, затем $u$ (или $v$), вернется в $w$,
        затем посетит $v$ (или $u$ соответственно). Затем опять посетит $w$.
        \item Можно посчитать порядок посещения вершин в dfs, и сохранить
        в массив Order (вершины записываем как при входе так и при
        возврате из детей, такой обход называется эйлеровым обходом).
        \item $First[u],First[v]$ – момент первого посещения вершин $u$ и $v$.
        \item $h[i]$ – высота вершины, которая сохранена в $Order[i]$.
        \item Задача сводится к $id = RMQ_h(First[u],First[v])$.
        \item Ответ лежит в $Order[id]$.
    \end{itemize}
    \subsubsection*{Сведение RMQ к LCA.}
    \begin{itemize}
        \item Будем использовать декартово дерево по неявному ключу.
        \item Корень - минимальный элемент (то есть значение элемента
        просто будет приоритетом в дереве).
        \item В каждом поддереве аналогично.
        \item $RMQ(l,r) = LCA(A[l],A[r])$.
    \end{itemize}

    \subsection{Понятие сети, потока в сети, разреза в сети. Остаточная сеть.}
    \label{subsec:flow_network}
    \begin{Def}
        Транспортная сеть $G = (V,E)$ представляет собой ориентированный
граф, в котором каждое ребро $(u,v) \in E$ имеет неотрицательную
пропускную способность (capacity), $c(u,v) > 0$, если ребро $(u,v) \notin E$,
то $c(u,v) = 0$. В транспортной сети выделяются две вершины:
источник (source) $s$ и сток (sink) $t$. Для удобства предполагается,
что каждая вершина лежит на неком пути из источника к стоку.
    \end{Def}
    \begin{Def}
        Потоком (flow) в $G$ является действительная функция
        $f : V\times V \rightarrow \mathbb{R}$, удовлетворяющая трем условиям:
        \begin{itemize}
            \item Ограничение пропускной способности (capacity constraint): $f(u, v) \leqslant c(u, v), \forall (u, v) \in V$.
            \item Антисимметричность (skew symmetry): $f(u, v) = -f(v, u), \forall (u, v) \in V$.
            \item Сохранение потока (flow conservation): $\sum\limits_{v \in V} f(u, v) = 0, \forall u \in V \setminus \{s, t\}$.
        \end{itemize}
    \end{Def}
    \begin{Def}
        Количество $f(u,v)$, которое может быть положительным, нулевым
или отрицательным, называется потоком (flow) из вершины $u$ в
вершину $v$. Величина потока определяется как:
        \[
        |f| = \sum\limits_{v \in V} f(s, v)
        \]
        т. е. это сумарный поток, выходящий из источника.
    \end{Def}
    \begin{Def}
        Суммарный положительный поток, входящий в вершину:
        \[
        \sum\limits_{v\in V, f(v,u) > 0} f(v, u).
        \]
    \end{Def}
    \begin{Def}
        Суммарный чистый поток в некоторой вершине равен разности
суммарного положительного потока, выходящего из данной
вершины, и суммарного положительного потока, входящего в нее.
Одна из интерпретаций свойства сохранения потока состоит в том,
что для отличной от источника и стока вершины входящий в нее
суммарный положительный поток должен быть равен выходящему
суммарному положительному потоку.
    \end{Def}
    \begin{Def}
        Пусть задана некая транспортная сеть $G = (V,E)$ с источником $s$ и
стоком $t$. Пусть $f$ – некоторый поток в $G$. Рассмотрим пару вершин
$u,v \in V$. Величина дополнительного потока, который мы можем
направить из $u$ в $v$, не превысив пропускную способность $c(u,v)$,
является остаточной пропускной способностью ребра $(u,v)$, и
задается формулой $c_f(u,v) = c(u,v)-f(u,v)$.
    \end{Def}

    \begin{lemma}
        Пусть $G = (V, E)$~-- транспортная сеть с источником $s$ и стоком $t$, а $f$~-- поток в $G$. Пусть $G_f$~-- остаточная сеть в $G$, порождённая потоком $f$, а $f'$~-- поток в $G_f$. Тогда сумма потоков $f + f'$ является потоком в $G$, и величина этого потока равна $|f + f'| = |f| + |f'|$.
    \end{lemma}
    \begin{Def}
        Для заданой транспортной сети $G = (V,E)$ и потока $f$, остаточной
        сетью в $G_f = (V,E_f)$, порожденной потоком $f$, является сеть $G_f = (V,E_f)$, где:
        $E_f = \{(u,v) \in V \times V : c_f(u,v) > 0\}$.
    \end{Def}
    \begin{Def}
        Для заданых транспортной сети $G = (V,E)$ и потока $f$ увеличивающим путем $p$ является простой путь из $s$ в $t$ в
остаточной сети $G_f$.
    \end{Def}
    \begin{Def}
        Максимальная величина, на которую можно увеличить поток вдоль
        каждого ребра увеличивающего пути $p$, называется остаточной
        пропускной способностью $p$ и задается формулой:
        $c_f(p) = \min\{c_f(u,v) : (u,v) \in p\}$
    \end{Def}
    \begin{Def}
        Разрезом $(S,T)$ транспортной сети $G = (V,E)$ называется разбиение
        множества вершин на множества $S,T = V-S$, такие что
        $s  \in S,t \in T$. Если $f$ – поток, то чистый поток через разрез равен:
        $f(S,T)$. Пропускной способностью разреза является $c(S,T)$.
        Минимальный разрез сети – разрез, пропускная способность
        которого среди всех разрезов минимальна.
    \end{Def}

    \subsection{Формулировка теоремы о максимальном потоке и минимальном разрезе (Форда-Фалкерсона). Время поиска максимального потока.}
    \label{subsec:ford_fulkerson}
    \begin{theorem}[Форда-Фалкерсона]
        Если $f$ – некоторый поток в транспортной сети $G = (V,E)$ с
источником $s$ и стоком $t$, то следующие утверждения
эквивалентны:
        \begin{itemize}
            \item $f$ – максимальный поток в G.
            \item Остаточная сеть не содержит увеличивающих путей.
            \item $|f| = c(S,T)$ для некоторого разреза $(S,T)$ сети $G$.
        \end{itemize}
    \end{theorem}
    \begin{lemma}
        Пусть $f$~-- некоторый поток в транспортной сети $G$ с источником $s$ и стоком $t$, и пусть $(S, T)$~-- разрез $G$. Тогда чистый поток через разрез равен $f(S, T) = |f|$.
    \end{lemma}
    \subsubsection*{Задача о максимальном потоке.}
    \begin{problem}
        Дана некоторая транспортная сеть $G$ с источником $s$ и стоком $t$, и
необходимо найти поток максимальной величины.
    \end{problem}
    \subsubsection*{Метод Форда-Фалкерсона.}
    \begin{itemize}
        \item Находим на каждой итерации увеличивающий путь и
        увеличиваем поток вдоль каждого ребра этого пути на
        величину остаточной пропускной способности $c_f(p)$. Пути ищем
        при помощи поиска в ширину.
        \item Не забываем об антисимметричности.
        \item Перед запуском алгоритма очевидно полагаем поток равным
        нулю.
    \end{itemize}
    \subsubsection*{Анализ работы.}
    \begin{itemize}
        \item Скорость работы зависит от метода поиска увеличивающих
        путей.
        \item В случае иррациональных чисел алгоритм может и не сойтись.
        \item Проанализируем работу алгоритма с целыми числами и
        использованием dfs для поиска увеличивающих путей.
        \item В случае целых чисел величина потока увеличивается всегда
        хотя бы на 1.
        \item Асимптотика $O(|E||f|)$.
    \end{itemize}
    \subsubsection*{Поиск паросочетаний при помощи потоков.}
    Хотим наибольшее паросочетание.
    \begin{itemize}
        \item Создадим фиктивный исток, и направим из него ребра веса 1 в
        каждую из вершин левой доли.
        \item Создадим фиктивный сток, в него направим ребра веса 1 из
        правой доли.
        \item Ориентируем все ребра из левой доли в правую, ставим вес 1.
        \item Ищем макс. поток.
    \end{itemize}

    \section{Вопросы на уд.}
    \subsection{Лемма о белых путях.}
    \begin{lemma}
        Рассмотрим момент, когда вершина v была покрашена в серый. Тогда
все вершины, достижимые по белым путям из v покрасятся в черный к
моменту выхода из v.
    \end{lemma}
    \begin{proof}
        \begin{itemize}
            \item Черные вершины, очевидно остануться черными.
            \item Серые вершины, очевидно остануться серыми, так как для того,
            чтобы они окрасились в черный, необходимо сначала чтобы вершина
            v окрасилась в черный (они все лежат в стеке рекурсии).
            \item Осталось показать что все достижимые из v белые вершины станут
            черными.
        \end{itemize}
        Заметим, что ни в какой момент времени не может быть такого, что есть
ребро из черной вершины в белую, это условие противоречит самому
алгоритму, ведь мы красим вершины в черный только тогда, когда все их
дети будут окрашены в черный (посещены). \\
Допустим $\exists u$, цвет которой отличен от черного и при этом она была
достижимой из v по белому пути.
        \begin{itemize}
            \item Если этот цвет — белый, то это значит что вершина u вообще не
            посещалась, если рассмотреть путь который был белым в момент
            когда v была серой, то можно заметить, после окраски v в черный у
            нас появилось ребро из черной вершины в белую, чего быть не
            может. Противоречие.
            \item Если этот цвет — серый, то эта значит, что мы исследовали не всех
            ее соседей, но так как мы посетили вершину u после вершины v, то
            из этого следует что мы не могли обработать и всех соседей
            вершины v. Противоречие.
        \end{itemize}
    Тогда вершина u не могла быть достижимой из v по белому пути.
    \end{proof}
    \subsection{Критерий двудольности графов. Алгоритм поиска разбиения на доли.}
    \begin{theorem}[Критерий двудольности графов]
        Граф $G = (V, E)$ является двудольным тогда и только тогда, когда
        его можно раскрасить в два цвета так, что никакие две смежные
        вершины не будут иметь одинаковый цвет. \\
        Эквивалентно: граф является двудольным тогда и только тогда, когда
        в нем нет циклов нечётной длины.
    \end{theorem}
    Проверить можно при помощи DFS или BFS.
    \subsection{Алгоритм Дейкстры.}
    См. \ref{subsubsec:decstra}.
    \subsection{Алгоритм Форда-Беллмана.}
    См. \ref{subsubsec:ford_bellman}.
    \subsection{Задача APSP. Алгоритм Флойда-Уоршелла.}
    \begin{problem}
        Решить задачу APSP, если $w : E \rightarrow \RR$.
        APSP — all pairs shortest paths.
    \end{problem}
    \begin{enumerate}
        \item Пусть $dp[u][v][k]$ равно минимальному весу пути из $u$ в $v$,
        состоящему из вершин с номерами $\leqslant k$.
        \item База. $dp[u][u][0] = 0$, $dp[:][:][:] = 0$.
        \item Переход. $dp[u][v][k] = \min(dp[u][v][k], dp[u][k][k-1] + dp[k][v][k-1])$.
        \item Порядок пересчета. Внешний цикл по $k$, внутри два вложенных цикла по $u$ и $v$.
        \item Ответ. $ans[u][v] = dp[u][v][|V|-1]$.
    \end{enumerate}
    Время: $O(|V|^3)$. Память: $O(|V|^3)$.

    \subsection{Постановка задачи поиска минимального остовного дерева. Лемма о безопасном ребре. Алгоритм Прима.}
    Cм. \ref{subsec:kruskal}.
    \subsection{Постановка задачи поиска наименьшего общего предка. LCA. Наивное решение. Решение с использованием двоичных подъемов.}
    См. \ref{subsec:lca}.
    \subsection{Алгоритм Куна для поиска максимального паросочетания в двудольном графе. Улучшения алгоритма Куна.}
    \label{subsec:kunas_algorithm}
    \begin{Def}
        Паросочетанием в неориентированном графе $G = (V,E)$ называют
    множество ребер $M \subseteq E$ такое, что не найдется двух ребер из $M$ с
    общей вершиной.
    \end{Def} 
    \begin{Def}
        Паросочетание $M$ называется максимальным, если не существует
паросочетания $M'$ такого, что $|M| < |M'|$. 
    \end{Def}
    \begin{Def}
        Увеличивающая цепь относительно паросочетания $M$ — путь
    $p = (v_1,\ldots,v_{2k} )$ такой, что $(v_1,v_2) \notin M, (v_2,v_3) \in M, (v_3,v_4) \notin M, \ldots,
    (v_{2k-1},v_{2k} ) \notin M$, при этом вершины $v_1$ и $v_{2k}$ не насыщены $M$.
    \end{Def}
    \begin{theorem}[Бержа]
        Паросочетание $M$ максимально тогда и только тогда, когда относительно $M$ нет увеличивающих цепей.
    \end{theorem}
    \begin{lemma}
        Если в графе степень каждой вершины не превосходит двух, то его ребра разбиваются на непересекающиеся пути и циклы.
    \end{lemma}
    \begin{lemma}
        Если из вершины v не была найдена увеличивающая цепь, то и далее из нее не найдется увеличивающая цепь.
    \end{lemma}
    \subsubsection*{Алгоритм Куна.}
    \begin{enumerate}
        \item Ищем увеличивающую цепь (при помощи dfs), чередуем ее.
        \item Сделаем так для всех вершин.
        \item Утверждается, что этого хватит.
    \end{enumerate}

    \subsubsection*{Поиск цепи.}
    Поиск увеличивающей цепи осуществляется с помощью
специального обхода в глубину. Изначально обход в глубину стоит в
текущей ненасыщенной вершине $v$ левой доли. Просматриваем все
рёбра из этой вершины, пусть текущее ребро $(v,to)$. Если вершина
$to$ ещё не насыщена паросочетанием, то, значит, мы смогли найти
увеличивающую цепь: она состоит из единственного ребра $(v,to)$; в
таком случае просто включаем это ребро в паросочетание и
прекращаем поиск увеличивающей цепи из вершины $v$. Иначе, если
$to$ уже насыщена каким-то ребром $(to,p)$, то попытаемся пройти
вдоль этого ребра: тем самым мы попробуем найти увеличивающую
цепь, проходящую через рёбра $(v,to)$, $(to,p)$. Для этого просто
перейдём в нашем обходе в вершину $p$ — теперь мы уже пробуем
найти увеличивающую цепь из этой вершины. \\

Таким образом, алгоритм Куна это n запусков DFS или его
сложность: $O(|V|(|V|+|E|))$.

    \subsubsection*{Улучшения алгоритма Куна.}
    \begin{itemize}
        \item Для запуска алгоритма лучше взять меньшую долю – тогда
        нужно меньше запусков.
        \item Можно проинициализировать исходное паросочетание
        каким-либо не максимальным паросочетанием, например
        просто жадно набрать ребер, это также потенциально
        уменьшит число запусков.
    \end{itemize}
    \subsection{Остаточная сеть. Дополняющий поток. Сложение потоков.}
    См. \ref{subsec:flow_network}.
    \subsection{Теорема «о максимальном потоке и минимальном разрезе» (Форда-Фалкерсона). Метод Форда-
    Фалкерсона. Пример долгой работы, при реализации через dfs. Сведение задачи поиска макси-
    мального паросочетания к задаче поиска максимального потока.}
    См. \ref{subsec:ford_fulkerson}.
    \subsection{Слоистая сеть. Блокирующий поток. Схема Диница. Число итераций в схеме Диница. Жадный
    поиск блокирующего потока.}
    \label{subsec:dinic}
    \begin{Def}
        Слоистой сетью $N_l$ для сети $N$ назовём $N_l = (G_l = (V, E_l), c_l, s, t)$, где
        \[
        E_l = \{e = (u, v) \in E \mid d[v] - d[u] = 1\},
        \]
        \[
        c_l(u, v) = c(u, v) \cdot \mathbf{I}\{(u, v) \in E_l\},
        \]
        $d[v]$~-- расстояние в рёбрах от $s$ до $v$.

    \end{Def}
    \begin{Def}
        Блокирующий поток — такой поток $f$, что не найдется пути, вдоль
которого поток из $s$ в $t$ можно увеличить.
    \end{Def}
    \subsubsection*{Схема Диница.}
    \begin{enumerate}
        \item Построим сеть $N$, определим поток $f=0$. Строим $N_f$.
        \item Строим слоистую сеть из остаточной $N_{f_l}$.
        \item Если в $N_{f_l}$, $t$ недостижима из $s$, то алгоритм окончен. Иначе найдем блокирующий поток $f'$ в $N_{f_l}$.
        \item $f=f+f'$, обновляем остаточную сеть вдоль пропущенного потока. 
    \end{enumerate}
    Очевидно, шаги 1, 2 и 4 делаются за $O(|V|+|E|)$. Осталось понять,
как искать блокирующий поток и сколько итераций будет, пока не
произойдет выход из алгоритма.
    \begin{prop}
        Расстояние между истоком и стоком строго увеличивается после
каждой фазы алгоритма, т.е. $d'[t]>d[t]$, где $d'[t]$ — значение,
полученное на следующей фазе алгоритма.
    \end{prop}
    \begin{proof}
        Проведём доказательство от противного. Пусть длина кратчайшего
пути из истока в сток останется неизменной после очередной фазы
алгоритма. Слоистая сеть строится по остаточной. Из
предположения следует, что в остаточной сети будут содержаться
только рёбра остаточной сети перед выполнением данной фазы,
либо обратные к ним. Из этого получаем, что нашёлся путь из $s$ в $t$,
который не содержит насыщенных рёбер и имеет ту же длину, что
и кратчайший путь. Но этот путь должен был быть
«заблокирован» блокирующим потоком, чего не произошло.
Получили противоречие. Значит длина изменилась.
    \end{proof}
    \begin{corollary}
        Число итераций в алгоритме Диница составляет $|V|-1$.
    \end{corollary}
    \subsubsection*{Удаляющий обход.}
    Идея заключается в том, чтобы по одному находить пути из истока
$s$ в сток $t$, пока это возможно. Обход в глубину найдёт все пути из
$s$ в $t$, если из $s$ достижима $t$, а пропускная способность каждого
ребра $c(u,v) > 0$, поэтому, насыщая рёбра, мы хотя бы единожды
достигнем стока $t$, следовательно блокирующий поток всегда найдётся. \\
Ускорим данный алгоритм. Будем удалять в процессе обхода в
глубину из графа все рёбра, вдоль которых не получится дойти до
стока $t$. Это очень легко реализовать: достаточно удалять ребро
после того, как мы просмотрели его в обходе в глубину (кроме того
случая, когда мы прошли вдоль ребра и нашли путь до стока). С
точки зрения реализации, надо просто поддерживать в списке
смежности каждой вершины указатель на первое не удалённое
ребро, и увеличивать этот указатель в цикле внутри обхода в
глубину. \\
Если обход в глубину достигает стока, насыщается как минимум
одно ребро, иначе как минимум один указатель продвигается
вперед. Значит один запуск обхода в глубину работает за
$O(|V|+K)$, где $K$ — число продвижения указателей. Ввиду того,
что всего запусков обхода в глубину в рамках поиска одного
блокирующего потока будет $O(P)$, где $P$ — число рёбер,
насыщенных этим блокирующим потоком, то весь алгоритм поиска
блокирующего потока отработает за $O(P|V|+\sum\limits_i K_i)$, что, учитывая,
что все указатели в сумме прошли расстояние $O(|E|)$, дает
асимптотику $O(P|V|+|E|)$. В худшем случае, когда блокирующий
поток насыщает все рёбра, асимптотика получается $O(|V||E|)$. \\
Таким образом, научились искать блокирующий поток за $O(|V||E|)$,
а значит алгоритм Диница с удаляющим обходом отработает за
$O(|V|^2|E|)$, что «на одну степень $|V|$ быстрее».


    \section{Вопросы на хор.}

    \subsection{Отношение сильной связности. Компоненты сильной связности. Алгоритм Косарайю.}
    Cм. \ref{subsec:strongly_connected_components}.
    \subsection{Отношение реберной двусвязности. Мосты. Функция $t_{up}$. Алгоритм поиска мостов.}
    Cм. \ref{subsec:edge_biconnectivity}.
    \subsection{Вершинная двусвязность. Точки сочленения. Функция $t_{up}$. Алгоритм поиска точек сочленения.}
    Cм. \ref{subsec:edge_biconnectivity}.
    \begin{Def}
        Две вершины вершинно двусвязны, если между ними есть два вершинно непересекающихся пути.
    \end{Def}
    \begin{Def}
        Точка сочленения — вершина, при удалении которой увеличивается число компонент связности.
    \end{Def}
    \begin{theorem}[Критерий точки сочленения]
        Рассмотрим древесное ребро (v,to).
        \begin{enumerate}
            \item Пусть $v$ — корень дерева обхода, тогда $v$ — точка сочленения $\iff$ у нее хотя бы два ребенка.
            \item Пусть $v$ — не корень дерева обхода, тогда $v$ — точка сочленения $\iff t_{up}(to) \geqslant t_{in}(v)$.
        \end{enumerate}
    \end{theorem}
    \begin{proof}
        \begin{enumerate}
            \item Это то же самое, что есть два древесных ребра из v. \\ 
            $\Rightarrow$ От противного. У $v$ всего один ребенок в дереве обхода. Но тогда
            удаление $v$ не ломает связности. \\
            $\Leftarrow$ К моменту обхода первого поддерева второе будет белым. Но в
            черные вершины ребер нет, значит нет ребер между этими
            поддеревьями, а единственный путь проходит через корень.
            \item $t_{up}(to) \geqslant t_{in}(v)$ значит, что выше $v$ из поддерева $to$ не прыгнешь. \\
            $\Rightarrow$ Очевидно. \\
            $\Leftarrow$ От противного. Пусть для всех детей $v$ верно, что
            $t_{up}(to) < t_{in}(v)$, тогда из каждого ребенка можно прыгнуть в
            наддерево, откуда $v$ — не точка сочленения. Противоречие.
        \end{enumerate}
    \end{proof}

    \subsection{Взвешенные графы. Обход в ширину BFS. Его модификации: 0-1 BFS, 1-k BFS, 0-k BFS.}
    См. \ref{subsec:bfs}.
    \subsection{Поиск циклов отрицательного веса.}
    См. \ref{subsubsec:ford_bellman_negative_cycle}.
    \subsection{Система непересекающихся множеств. Реализация с использованием леса деревьев. Эвристики. Время работы (б/д).}
    Cм. \ref{subsec:DSU}
    \subsection{Постановка задачи поиска минимального остовного дерева. Лемма о безопасном ребре. Алгоритм Крускала.}
    Cм. \ref{subsec:kruskal}.
    \subsection{Алгоритм Борувки.}
        Имеем связный граф $G = (V,E)$, с заданной весовой функцией
    $w : E \rightarrow \mathbb{R}$. Хотим найти минимальное остовное дерево. Мы уже
    знаем два алгоритма, которые с этим хорошо справляются:
    \begin{itemize}
        \item Алгоритм Крускала.
        \item Алгоритм Прима.
    \end{itemize}
    Познакомимся с еще одним замечательным алгоритмом поиска
    минимального остовного дерева. Для этого нам понадобится одна
    простая лемма.
    \begin{lemma}
        Для любой вершины наименьшее инцидентное к ней ребро является безопасным.
    \end{lemma}
    \begin{proof}
        Рассмотрим некоторую вершину $u$, обозначим наименьшее инцидентное ей ребро как $(u, v)$. Предположим, имеется $T_{\text{min}} \subset E$, являющийся минимальным остовным деревом, такое, что $(u, v) \notin T_{\text{min}}$. Поскольку остовное дерево связно, существует путь $P$ из $u$ в $v$. Покажем, что такое дерево можно улучшить:

У вершины $u$ существует инцидентное ребро $(u, z) \in T_{\text{min}}$. Добавим $(u, v)$ в $T_{\text{min}}$. Это ребро представляет из себя путь из $u$ в $v$. Таким образом, мы образовали цикл. Можно удалить $(u, z)$ и получить дерево $T_{\text{min}}^2$.

В силу минимальности веса $(u, v)$: $w(T_{\text{min}}^2) \leq w(T_{\text{min}})$. Противоречие.

    \end{proof}

    Алгоритм Борувки:
    \begin{enumerate}
        \item Имеем лес деревьев, изначально это лес из вершин графа.
        \item Для каждого дерева найдем наименьшее инцидентное дереву
        ребро.
        \item Добавим эти ребра к деревьям (некоторые могут добавиться
        более одного раза).
        \item После этого некоторые деревья склеиваются в новые деревья,
        более крупные.
        \item Повторяем шаги 2-4, пока не останется одно дерево.
    \end{enumerate}
    \subsubsection*{Сложность алгоритма Борувки.}
    \begin{itemize}
        \item Временная сложность алгоритма составляет $O(|E| \log |V|)$.
        \item На каждом шаге число деревьев уменьшается хотя бы в два
        раза, так как каждое дерево задействовано в объединении.
        \item Потребление памяти составляет $O(|V|)$.
        \item Нам нужно поддерживать СНМ, а также минимумы, СНМ
        содержит не более чем $V$ множеств.
    \end{itemize}
    \subsection{Постановка задачи поиска наименьшего общего предка LCA. Сведение LCA $\iff$ RMQ.}
    См. \ref{subsec:lca}.
    \subsection{Двудольность графов. Паросочетания. Увеличивающие цепи и теорема Бержа.}
    Cм. \ref{subsec:kunas_algorithm}.
    \subsection{Поток через разрез. Неравенство между величиной произвольного потока и пропускной спо-
    собности произвольного разреза.}
    См. \ref{subsec:flow_network}.
    \subsection{Алгоритм Эдмондса-Карпа. Время работы.}
    В алгоритме Эдмондса-Карпа мы запускаем BFS для поиска увеличивающего пути от истока $s$ к стоку $t$, а затем пропускаем по найденному пути весь возможный поток, равный минимальной остаточной пропускной способности на этом пути.
    \begin{theorem}
        Алгоритм Эдмондса-Карпа работает за $O(|V||E|^2)$.
    \end{theorem}
    \begin{lemma}
        Если в сети $N = (G,c,s,t)$ увеличение потока производится вдоль
кратчайших $s\rightsquigarrow t$ путей в $N_f$, то $\forall v \in V \setminus \{s,t\}$ длина кратчайшего пути $d_f(s,v)$ в $N_f$ не убывает.
    \end{lemma}
    \begin{proof}
        Пусть $f, f'$~-- потоки в $N$, между которыми одна итерация. Пусть $v$~-- вершина, $d_f(s, v)$ до которой минимально и $d_{f'}(s, v) < d_f(s, v)$. Рассмотрим путь $p = s \rightsquigarrow u \to v$, являющийся кратчайшим от $s$ до $v$ в $N_{f'}$. Тогда верно, что $d_{f'}(s, u) = d_{f'}(s, v) - 1$.

По выбору $v$ и из предыдущего утверждения получаем, что
$d_{f'}(s, u) \geq d_f(s, u)$.
        \begin{itemize}
            \item Если $(u, v) \in E_f$, тогда
            \[
            d_f(s, v) \leq d_f(s, u) + 1 \leq d_{f'}(s, u) + 1 = d_{f'}(s, v).
            \]
            
            \item Если $(u, v) \notin E_f$, но $(u, v) \in E_{f'}$. Появление $(u, v)$ означает увеличение потока по обратному ребру $(v, u)$. Увеличение потока производится вдоль кратчайшего пути, поэтому кратчайший путь из $s$ в $u$, вдоль которого происходило увеличение, выглядит как $s \rightsquigarrow v \to u$, откуда
            \[
            d_f(s, v) = d_f(s, u) - 1 \leq d_{f'}(s, u) - 1 = d_{f'}(s, v) - 2.
            \]            
        \end{itemize}
    \end{proof}
    \begin{proof}
        Назовём ребро $e$ вдоль кратчайшего пути $p$ в $N_f$ от $s$ до $t$ критическим, если $c_f(e) = \min\limits_{e \in p} c_f(p)$. Покажем, что каждое ребро становится критическим $O(|V|)$ раз.

Заметим, что после увеличения все критические рёбра исчезают из остаточной сети. Рассмотрим $(u, v) \in E$. Увеличение производится вдоль кратчайших путей, поэтому если $(u, v)$ становится критическим в первый раз, то $d_f(s, v) = d_f(s, u) + 1$. Затем оно исчезает из сети и не появится, пока не будет уменьшено по обратному ребру $(v, u)$.

Пусть в момент перед увеличением поток в сети $N$ составлял $f'$, тогда $d_{f'}(s, u) = d_{f'}(s, v) + 1$. Согласно лемме $d_f(s, v) \leq d_{f'}(s, v)$, откуда
\[
d_{f'}(s, u) = d_{f'}(s, v) + 1 \geq d_f(s, v) + 1 = d_f(s, u) + 2.
\]
Получаем, что между итерациями, когда $(u, v) \in E$ становится критическим, расстояние от $s$ до $u$ увеличивается на 2, откуда $O(|V|)$ раз оно могло становиться критическим.

Всего рёбер $O(|E|)$, значит, суммарное число итераций составит $O(|V||E|)$.

Время работы каждой итерации~-- $O(|E|)$.

Тогда итоговое время работы алгоритма Эдмондса-Карпа: $O(|V||E|^2)$.

    \end{proof}
    \subsection{Слоистая сеть. Блокирующий поток. Схема Диница. Число итераций в схеме Диница. Удаляющий обход.}
    См. \ref{subsec:dinic}
    \subsection{Алгоритм масштабирования потока.}

    \begin{itemize}
        \item Введем $U=\max\limits_{(u, v)\in E} c(u, v)$.
        \item Рассмотрим $k=\lfloor \log_2 U\rfloor \ldots 0, \Delta_k=2^k$.
        \item На каждой фазе рассматриваем в остаточной сети только
        ребра с пропускными способностями хотя бы $\Delta_k$ и более.
        Таким образом мы будем сначала пытаться протолкнуть
        большие потоки и только потом переходить к маленьким.
        \item $\Delta_0=1$, при нем алгоритм вырождается в алгоритм
        Эдмондса-Карпа, в следствии чего он является корректным.
    \end{itemize}
    \begin{lemma}
        $|f|$ в сети $G$ ограничен сверху значением $|f_k|+2^k|E|$.
    \end{lemma}
    \begin{proof}
        Рассмотрим разрез. Вспомним что поток ограничен пропускной
способностью разреза. Поскольку на $k$ фазе мы рассматриваем
ребра с пропускными способностями от $2^k$, то остаточная
возможная пропускная способность будет не более чем $2^k|E|$, в
свою очередь $f_k$ — наденый поток на фазе $k$.
    \end{proof}
    \begin{lemma}
        Время работы алгоритма Эдмондса-Карпа с использованием
техники масштабирования потока составляет $O(|E|^2\log U)$.
    \end{lemma}
    \begin{proof}
        \begin{itemize}
            \item Всего фаз у алгоритма $\log U$.
            \item Поиск увеличивающего пути с помощью поиска в ширину
            работает за $O(|E|)$.
            \item На каждой фазе у нас не более чем $2|E|$ увеличивающих путей,
            так как на поток на предыдущей фазе ограничен $2^{k+1}|E|$, а
            каждый увеличивающий путь имеет пропускную способность
            как минимум $2^k$.
        \end{itemize}
        Объединив эти три факта имеем, что время работы алгоритма
составляет $O(|E|^2\log U)$. Отметим, что использование техники
оправдано, когда $\log U < |V|$.
    \end{proof}
    \subsection{Теорема о декомпозиции потока.}
    \begin{theorem}[О декомпозиции]
        Любой поток $f$ в сети $G$ можно представить в виде:
        \begin{itemize}
            \item Набора $s-t$ путей $P_1,\ldots, P_k$ с потоками $f_1,\ldots,f_k > 0$
            \item Набора циклов $C_1,\ldots,C_m$ с потоками $f_{k+1},\ldots,f_{k+m} > 0$
        \end{itemize}
        При этом:
        \begin{enumerate}
            \item $f(e)=\sum\limits_{i:e\in P_i} f_i + \sum\limits_{j: e\in C_j} f_{k+j}$
            \item Суммарный поток путей равен величине потока: $\sum\limits_{i=1}^k f_i=|f|$ 
        \end{enumerate}
    \end{theorem}
    Алгоритм построения декомпозиции:
    \begin{enumerate}
        \item Пока в сети есть ненулевой поток из s:
        \begin{enumerate}
            \item Начинаем из $s$, выбираем ребро с $f(e) > 0$ в $v_1$
            \item Если $v_1 = t$: найден $s$–$t$ путь $P_i$
            \item Иначе, по сохранению потока, $\exists$ ребро из $v_1$ в $v_2$ с $f > 0$
            \item Продолжаем, пока не попадём в $t$ (путь) или в посещённую вершину (цикл)
            \item Находим минимальный поток $f_i$ по рёбрам пути/цикла
            \item Добавляем путь/цикл в декомпозицию с потоком $f_i$
            \item Вычитаем $f_i$ из $f$ по всем рёбрам пути/цикла
            \item Повторяем для оставшихся ненулевых потоков (циклы)
          \end{enumerate}
        \item Повторяем для оставшихся ненулевых потоков (циклы).
    \end{enumerate}
    Корректность:
    \begin{enumerate}
        \item На каждом шаге уменьшается поток хотя бы по одному ребру.
        \item Максимальное число операций — $|E|$ (по числу рёбер).
        \item Сохранение потока гарантирует возможность продолжения.
        \item Величина $|f|$ уменьшается на поток каждого $s$–$t$ пути.
        \item После обнуления потока из $s$ остаются только циклы.
    \end{enumerate}
    Завершение:
    \begin{enumerate}
        \item Когда все потоки нулевые, декомпозиция построена.
        \item Сумма потоков $s$–$t$ путей равна исходному $|f|$.
        \item Циклы не влияют на величину потока.
    \end{enumerate}
    \begin{prop}
        Время работы алгоритма декомпозиции потока составляет $O(|V||E|)$.
    \end{prop}
    \begin{proof}
        \begin{itemize}
            \item Каждый путь или цикл содержит не более $|V|$ вершин $\Rightarrow$ поиск одного пути/цикла занимает $O(V)$.
            
            \item Декомпозиция содержит не более $|E|$ путей/циклов:
            \begin{itemize}
              \item На каждом шаге обнуляется хотя бы одно ребро.
              \item Всего рёбер — $|E|$.
            \end{itemize}
            
            \item Каждая вершина рассматривается только при наличии ненулевого потока через неё.
            \item Итого: $O(V)$ на путь $\times O(E)$ путей $= O(|V||E|)$.
          \end{itemize}
    \end{proof}
    \subsection{Биномиальная куча.}
    \begin{Def}
        Биномиальное дерево $B_k$ — дерево, определяемое для каждого
$k = 0,1,2,\ldots$ следующим образом: $B_0$ — дерево, состоящее из одного
узла; $B_k$ состоит из двух биномиальных деревьев $B_{k-1}$, связанных
вместе таким образом, что корень одного из них является дочерним
узлом корня второго дерева.
    \end{Def}
    \begin{Def}
        Биномиальная пирамида $H$ – представляет из себя множество
биномиальных деревьев, которые удовлетвораяют следующим
свойствам:
        \begin{enumerate}
            \item Каждое биномиальное дерево в $H$ подчиняется свойству
            неубывающей пирамиды: ключ узла не меньше ключа его
            родительского узла. Мы говорим, что такие деревья являются
            упорядоченными в соответствии со свойством неубывающей
            пирамиды.
            \item Для любого неотрицательного целого $k$ имеется не более
            одного биномиального дерева $H$, чей корень имеет степень $k$.
        \end{enumerate}
        Таким образом каждое дерево содержит самый маленький элемент.
    \end{Def}
    \subsubsection*{Хранение.}
    \begin{itemize}
        \item В списке корней храним деревья в порядке строгого
        возрастания степеней корней.
        \item next\_tree имеет различный смысл для корней и для детей, для
        корней – это следующий корень в списке, для детей –
        следующий ребенок.
    \end{itemize}
    \subsubsection*{Поиск минимума.}
    \begin{itemize}
        \item Для поиска минимума достаточно найти наименьший корень.
        \item Функция будет работать за $O(\log N)$.
    \end{itemize}
    \subsubsection*{Слияние.}
    \begin{itemize}
        \item При слиянии двух куч сначала сольем их корневые списки, так
        чтобы степени корней неубывали.
        \item Далее будем сливать соседние вершины одинаковых степеней.
        \item Важно что объединении двух деревьев может образоваться три
        дерева одной степени подряд. Этот случай важно правильно
        обработать, пропустив первое из трех деревьев.
    \end{itemize}
    \subsubsection*{Вставка.}
    \begin{itemize}
        \item Создаем кучу из одного биномиального дерева.
        \item Эту кучу можно слить с исходной.
        \item Работать это будет за $O(\log n)$.
    \end{itemize}
    \subsubsection*{Извлечение минимума.}
    \begin{itemize}
        \item Ищем min корень в списке корней.
        \item Заметим что его дети – это тоже биномиальные деревья.
        \item Нужно перевернуть список детей, и это будет биномиальной
        кучей.
        \item Сольем две кучи.
    \end{itemize}
    \subsection{Постановка задачи о поиске потока минимальной стоимости. Задача о назначениях и ее сведение к задаче о поиске потока минимальной стоимости.}
    \begin{Def}
        Взвешенной сетью назовем пару $(N ,w)$, где $w : E\rightarrow R$, при этом
нет циклов отрицательного веса.
    \end{Def}
    \begin{Def}
        Стоимость потока — величина, равная
        \[\sum\limits_{(u, v)\in E(N)} f(u, v)w(u,v)\]
    \end{Def}
    \begin{problem}
        Необходимо найти максимальный поток при этом с минимальной
суммарной стоимостью (min cost max flow) или поток величины k
минимальной стоимости (min cost k-flow).
    \end{problem}
    \begin{prop}
        Пусть $f$ — максимальный поток. В $N_f$ нет циклов отрицательного
веса $\iff f$ имеет минимальную стоимость.
    \end{prop}
    \begin{proof}
        $\Rightarrow$ Пусть $f$ неоптимальный и $f^*$ оптимальный (по стоимости). $f^*-f$ раскладывается в набор простых циклов и $|f^*-f|<0$, значит есть цикл отрицательного веса. \\
        $\Leftarrow$ Пропустим вдоль цикла отрицательного веса поток, чтобы его убрать. Величина не изменится, а стоимость уменьшится.
    \end{proof}
    \subsubsection*{Задача о назначениях.}
    \begin{problem}
        \begin{itemize}
            \item Имеется $N$ заказов и $N$ станков
            \item Для каждого заказа известна стоимость изготовления на
            каждом станке ($A_{ij}$).
            \item Каждый станок может выполнять только один заказ
        \end{itemize}
        Найти распределение заказов по станкам, минимизирующее
суммарную стоимость:
        \[\min_{\sigma \in S_N} \sum_{i=1}^N A_{i, \sigma(i)}\]
        где $S_N$ - множество всех перестановок порядка $N$.
    \end{problem}
    \subsubsection*{Сведение задачи к поиску потока минимальной стоимости.}
    \begin{enumerate}
    \item Строим ориентированный граф $G$:
    \begin{itemize}
        \item Исток $S$ и сток $T$;
        \item $N$ вершин для заказов (первая доля);
        \item $N$ вершин для станков (вторая доля).
    \end{itemize}
    
    \item Добавляем рёбра:
    \begin{itemize}
        \item Из $S$ в каждую вершину-заказ: пропускная способность $1$, стоимость $0$;
        \item Из заказов в станки: пропускная способность $1$, стоимость $A_{ij}$;
        \item Из каждой вершины-станка в $T$: пропускная способность $1$, стоимость $0$.
    \end{itemize}
    
    \item Находим максимальный поток минимальной стоимости.
    \end{enumerate}
\end{document}

